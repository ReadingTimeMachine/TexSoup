\documentclass[a4paper]{article}

\usepackage{hyperref}

\newcommand{\be}{\begin{ equation}}
\newcommand{\ee}{\end{equation }}

\newcommand{\bee}[1]{\begin{equation}\label{e:#1}}
\newcommand{\eee}{\end{equation}}

\newcommand{\beee}[2]{\begin{equation}\label{e:#1} #2}
\newcommand{\eeee}{\end{equation}}


\def \etal {{\em et al.}}

\begin{document}

\section{Waffles waffles waffles}

\subsection{Waffles are cooler than you}

Waffles have significantly more pockets than you.  

\[
P(W) > P(\rm{you})
\]

Ok, maybe you are at least as cool as waffles:

$$
C_{ool}(\rm{you}) \ge C_{ool}(W)
$$

Sometimes though, you might type things in silly ways $C_ {ool  } (\rm{anybody}) >   0$.

\begin  {equation}
C_{   ool} \ne 0
\end{equation }

Also, let's do a silly redefined an equation:
\be
C = C_2 + 1
\ee

Also also, just one more time:
\be
C = C_2 + 1
\ee


Sometimes I like to write with weird commands like Naiman \etal\ which can make it tricky to parse things.

\bee{lenseq}
    A(\xi) = \frac12\left(\xi+\frac1\xi\right),
\eee

\beee{lenseq2}{A(x)=5}\eeee

Getting some refs in \ref{lenseq} and \ref{lenseq2}.

Let me also try out s\`ome accents in d{\"{i}}ff\'e\'r\`ent f\'{o}rms. Doin\`{g} it.

\'Ok great!

Well that was fun!


\end{document}
